\documentclass[fancyhdr,bookmark]{ctexbook}
\setCJKmainfont{SimSun}
\setmainfont{Georgia} 	% 設定英文字型
\setromanfont{Georgia} 	% 字型
\setmonofont{Latin Modern Mono}

% pandoc版本大于1.15时需要\tightlist
\providecommand{\tightlist}{%
  \setlength{\itemsep}{0pt}\setlength{\parskip}{0pt}}

\usepackage{tikz} % Required for drawing custom shapes
\usepackage[yyyymmdd,hhmmss]{datetime}
\ctexset{today=small}
\usepackage{geometry} 		% 設定邊界
\geometry{
  top=1in,
  inner=1in,
  outer=1in,
  bottom=1in,
  headheight=3ex,
  headsep=2ex
}


\usepackage{xcolor}
\definecolor{ocre}{RGB}{243,102,25} % Define the orange color used for highlighting throughout the book

%\usepackage[x11names,svgnames,dvipsnames]{xcolor}
\usepackage{listings}
\lstset{
	%numbers=left,
	%numberstyle=\tiny,
	basicstyle=\small\ttfamily,
	keywordstyle=\color[rgb]{0.13,0.29,0.53}\textbf,
	commentstyle=\color[rgb]{0.56,0.35,0.01}\textit,
	identifierstyle=\color[rgb]{0.00,0.00,0.00},
	stringstyle=\color[rgb]{0.31,0.60,0.02},
	frame=shadowbox,
	rulesepcolor=\color{red!20!green!20!blue!20},
	backgroundcolor=\color[rgb]{0.97,0.97,0.97},
	tabsize=4,
	breaklines=tr,
	showstringspaces=false,
}
\renewcommand{\lstlistingname}{代码}

%\newcommand{\KeywordTok}[1]{\textcolor[rgb]{0.13,0.29,0.53}{\textbf{{#1}}}}
%\newcommand{\DataTypeTok}[1]{\textcolor[rgb]{0.13,0.29,0.53}{{#1}}}
%\newcommand{\DecValTok}[1]{\textcolor[rgb]{0.00,0.00,0.81}{{#1}}}
%\newcommand{\BaseNTok}[1]{\textcolor[rgb]{0.00,0.00,0.81}{{#1}}}
%\newcommand{\FloatTok}[1]{\textcolor[rgb]{0.00,0.00,0.81}{{#1}}}
%\newcommand{\CharTok}[1]{\textcolor[rgb]{0.31,0.60,0.02}{{#1}}}
%\newcommand{\StringTok}[1]{\textcolor[rgb]{0.31,0.60,0.02}{{#1}}}
%\newcommand{\CommentTok}[1]{\textcolor[rgb]{0.56,0.35,0.01}{\textit{{#1}}}}
%\newcommand{\OtherTok}[1]{\textcolor[rgb]{0.56,0.35,0.01}{{#1}}}
%\newcommand{\AlertTok}[1]{\textcolor[rgb]{0.94,0.16,0.16}{{#1}}}
%\newcommand{\FunctionTok}[1]{\textcolor[rgb]{0.00,0.00,0.00}{{#1}}}
%\newcommand{\RegionMarkerTok}[1]{{#1}}
%\newcommand{\ErrorTok}[1]{\textbf{{#1}}}
%\newcommand{\NormalTok}[1]{{#1}}



\ifxetex
  \usepackage[setpagesize=false, % page size defined by xetex
              unicode=false, % unicode breaks when used with xetex
              xetex]{hyperref}
\else
  \usepackage[unicode=true]{hyperref}
\fi
\hypersetup{breaklinks=true,
            bookmarks=true,
            pdfauthor={An He},
            pdftitle={iTop实施杂记},
            colorlinks=true,
            urlcolor=blue,
            linkcolor=magenta,
            pdfborder={0 0 0}}
\urlstyle{same}  % don't use monospace font for urls
% Make links footnotes instead of hotlinks:
\renewcommand{\href}[2]{#2\footnote{\url{#1}}}



\title{iTop实施杂记}
\author{An He}
\date{\today}

\usepackage{fancyhdr}
%\usepackage{lastpage}
\pagestyle{fancy}


\begin{document}
\frontmatter
%----------------------------------------------------------------------------------------
%	TITLE PAGE
%----------------------------------------------------------------------------------------

\begingroup
\thispagestyle{empty}
\begin{tikzpicture}[remember picture,overlay]
\node[inner sep=0pt] (background) at (current page.center) {\includegraphics[width=\paperwidth]{Pictures/background}};
\draw (current page.center) node [fill=ocre!30!white,fill opacity=0.6,text opacity=1,inner sep=1cm]
{\Huge\centering\bfseries\sffamily\parbox[c][][t]{\paperwidth}
{\centering iTop实施杂记\\[13pt] % Book title
{\huge An He} % Author name
}};
\end{tikzpicture}
\vfill
\endgroup
\addcontentsline{toc}{chapter}{封面}

%----------------------------------------------------------------------------------------
%	COPYRIGHT PAGE
%----------------------------------------------------------------------------------------
\newpage
~\vfill
\thispagestyle{empty}

\noindent Copyright \copyright\ \the\year\  An He\\ % Copyright notice

\noindent \textsc{Published by \LaTeX}\\ % Publisher
\noindent \textsc{https://github.com/annProg/itop-book}\\ % URL

\noindent Licensed under the Creative Commons Attribution-NonCommercial 3.0 Unported License (the ``License''). You may not use this file except in compliance with the License. You may obtain a copy of the License at \url{http://creativecommons.org/licenses/by-nc/3.0}. Unless required by applicable law or agreed to in writing, software distributed under the License is distributed on an \textsc{``as is'' basis, without warranties or conditions of any kind}, either express or implied. See the License for the specific language governing permissions and limitations under the License.\\ % License information

\noindent \textit{最后编译日期, \today\ \currenttime } % Printing/edition date


    
\chapter*{前言}
\addcontentsline{toc}{chapter}{前言}
笔者是互联网运维行业从业者,2014年毕业后就进入一个知名互联网公司做基础运维工作,
当时是很兴奋的,然而工作没几天就蔫了。每天做的事不外乎处理研发部门提过来的故障工单,给
机房外包发重启重装工单等等。最讨厌的是改IP,改完之后还要手动去CMDB更新IP以及上联交换机信息。
总之就是复制粘贴。修修补补写了几个半自动化的脚本,依然感觉很无力。

第二份工作进入一个规模较大研发部门做运维。部门业务最小单元称为APP,APP会使用域名、
数据库等资源。部门有400多个APP,近1000的域名,1000多台服务器,以及数百个Mongo,MySQL
数据库。最直接的是问题仅凭记忆搞不清楚资源的归属,经常要在部门大群里问,这个域名是谁负责
啊?这个数据库是我们的业务在用吗?这个APP报警了,zabbix里没设置联系人,谁负责啊?如何
有效的管控这些资源是个问题。

另外一个比较鸡肋的工作是监控报警的维护工作。报警以APP为最小单元,APP负责人的变化是比较频繁的,
业务调整,人员离职都会有负责人的变化。因此经常要做的事情是:业务交接人员变更后,去
zabbix改报警接收人。对于Url监控,需要将研发人员提交的监控配置文件手动加入监控。
还有一个讨厌的事情,充当HUB代研发人员向集团运维部门提交资源申请工单。还是复制粘贴的活。
非常希望有一个自动化的手段来处理这些事情。

针对这些情况,笔者调研了几个开源CMDB系统,包括yourcmdb,i-doit等。最终选择了文档功能都较为
完善,开发也较为活跃的iTop作为CMDB工具进行定制,并且实现了以下功能:APP,数据库,服务器,
域名等资源入库;人员与APP关联,APP与资源的关联,开放人员关联查询的接口给zabbix,避免用
zabbix维护报警联系人;定制iTop权限让研发人员可以编辑自己负责的APP,相当于自助订阅报警;
将iTop作为Url监控配置的前端表单,实现研发自助添加Url监控;
定制工单系统,自动指派工单,自动录入新申请资源。
实施之后,之前混乱低效的情况有所好转。但是距离理想状态还差很远,理想状态是我这个职位就
不该存在,公司层面的运维提供良好的资源管控与监控报警系统,研发人员直接操作资源与报警,就像各
种云计算平台那样,这样一来,我这个中间层HUB就可以失业了,运维的效率也可以提高了。

遗憾的是,公司层面并没有这样的平台。随着公司爆发的危机,才发现我这个职位作为一个部门
的对接人在这种混乱环境里可能还少不了,否则可能会更混乱。为什么这么说呢?因为公司缺少一个
标准化运维的体系。就拿业务线来说,没有标准统一的业务线名称(工单系统申请资源的时候业务
线那一栏是文本,随便填),很多时候要靠部门来区分资源归属,这样确实需要一个部门接口人。
数据库、域名等资源没有纳入公司CMDB,也没有其他有效的管理方案,哪些在用,哪些下线,哪个
业务在用,联系人是谁等等都不清晰。多次用群发Excel表格这种低效又没准头的手段来统计使用
信息。梳理域名时发现我部门名下有1200多个域名,很多我没见过也ping不通的域名在列表里。
数据库也有大量已经不再使用的。服务器更是有相当一部分的闲置。这种混乱带来的不仅仅是效率
低下,还有高昂的运维成本以及负反馈带来的恶性循环:资源无法得到有效的管控,不能及时回收,
需要不断的购入新的服务器资源,最终只会越来越混乱。危机爆发后,在付不出钱的情况下,经常
整个机房的关停,业务方也要中断正常工作忙着迁移业务,最严重的一次机房在没有通知的情况下
直接断电,核心业务中断几个小时。

可以想见,如果有一个高效的标准化运维体系,有效的管控资源,即使在危机爆发的情况下,也不至
于那样狼狈。这些经历,使我看到CMDB对IT组织的必要性,也感受到CMDB实施是一个范围很大也很复
杂的工程。笔者从业时间不长,经验不丰富,对ITSM等理论也不熟悉。因此不打算从理论
上探讨CMDB,只从iTop的实施经历,谈谈iTop
CMDB的定制与运维自动化的一些经验。

本书计划安排以下内容: \textbf{第一章}
简单介绍一下iTop,iTop的插件开发流程,CI定制,本地化等等 \textbf{第二章}
CI属性约束,唯一性、profile权限,只读,隐藏等等 \textbf{第三章}
SSO集成方法 \textbf{第四章} iTop简介 iTop插件开发流程
CI定制(删除、新增、修改) menunode定制 本地化
CI属性约束(唯一性,只读,隐藏) SSO集成方法 action-shell-exec trigger
CI生命周期(lifecycle) cmdbapi rest扩展 api-client
request-template,工单自动指派,资源自动入库 custom-pages iframe嵌入
AttributeClassCustom 其他常用插件介绍 审计 Profile,自助化与权限控制
实例(URL监控) api相关的可以放在系统集成部分


{
\hypersetup{linkcolor=black}
\setcounter{tocdepth}{2}
\tableofcontents
\addcontentsline{toc}{chapter}{目录}
}
\listoftables
\addcontentsline{toc}{chapter}{表格列表}
\listoffigures
\addcontentsline{toc}{chapter}{插图列表}



\mainmatter
    % 在此命令之后的页码为阿拉伯数字
    % 以下为正文
\chapter{SSO集成方法}\label{ssoux96c6ux6210ux65b9ux6cd5}

\chapter{CI校验}\label{ciux6821ux9a8c}

\chapter{action-shell-exec}\label{action-shell-exec}

\backmatter

\end{document}
